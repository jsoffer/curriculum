%% start of file `template.tex'.
%% Copyright 2006-2011 Xavier Danaux (xdanaux@gmail.com).
%
% This work may be distributed and/or modified under the
% conditions of the LaTeX Project Public License version 1.3c,
% available at http://www.latex-project.org/lppl/.

\documentclass[11pt,letter,sans]{moderncv}

\moderncvstyle{casual}
\moderncvcolor{blue}

% character encoding
\usepackage[utf8]{inputenc}                   % replace by the encoding you are using

% adjust the page margins
\usepackage[scale=0.8]{geometry}
%\setlength{\hintscolumnwidth}{3cm}           % if you want to change the width of the column with the dates


% personal data
\firstname{Jaime}
\familyname{Soffer}
\address{Guelatao de Juárez, Iztapalapa}{09229 México, D.F.}
\mobile{~552~265~1295}
\email{jaime.soffer@gmail.com}
\homepage{http://www.soffernet.com/jaime/cms/}
\extrainfo{Jaime Alejandro Soffer Hernández}

%----------------------------------------------------------------------------------
%            content
%----------------------------------------------------------------------------------

\begin{document}
\maketitle

\section{Experiencia Profesional}

\cventry {May 2012--Oct 2013}
        {Ingeniero de datos y Gestor de proyecto}
        {Fractal Labs, Fractal Media}
        {México, D.F}{}{}
        \cvitem{}{DARTS (Digital Advertising Research \& Tracking Services)}
        \cvitem{}{La operación del servicio consistió en cuatro fases: adquisición
        de información estadística y medatatos de anuncios aparecidos en los sitios
        de internet más visitados en México, almacenamiento de dicha información,
        clasificación asistida por personal, y generación de reportes. }
        \cvitem{dirección}{ Andrés Gómez Urquiza }
        \cvitem{funciones en adquisición}{
            \begin{itemize}
                \item Navegación automática y parseo HTML en Javascript (PhantomJS)
                \item Navegación automática y parseo HTML en Ruby (Nokogiri; Watir - prototipo)
                \item Adquisición y parseo de metadatos en resultados de búsqueda (NodeJS)
                \item Almacenamiento de tráfico web por archivo, incluyendo actividad de Flash, vía un proxy especializado (Ruby)
            \end{itemize}
        }
            \cvitem{en almacenamiento}{
            \begin{itemize}
            \item Optimización del modelo de datos MySQL para aprox. 6000 escrituras por hora, generación de reportes mensuales en menos de dos minutos, un año de entradas
            \item Funciones de monitoreo de frecuencia de ingreso de datos (Python, SQL)
            \end{itemize}
        }
            \cvitem{en clasificación}{
            \begin{itemize}
            \item Depuración del sistema de clasificación manual de anuncios (Python, Django)
            \item Sistema de corrección de entradas basado en CSV, que verifica integridad y consistencia (Python, SQL)
            \item Supervisión de dos empleados a cargo de la clasificación
            \item Generación de reportes temporales en HTML para que los clasificadores verifiquen trabajo parcial, y para control de calidad (Python)
            \item
            \end{itemize}
        }
            \cvitem{en reportes}{
            \begin{itemize}
                \item Reducción del tiempo de generación de reporte de horas a minutos (MySQL - era Clojure)
                \item Implementación de campos adicionales al reporte (SQL - Agrupamientos, aritmética)
                \item Depuración de interfaz de presentación de reporte (Ruby, Rails)
                \item Sistema de consolidación de capturas de pantalla y capturas de plugin Flash (shell, Python)
            \item
            \end{itemize}
        }

\cventry {2011, Oct--Nov}
        {Programación Web}
        {Laboratorio de Informática Médica, UAM-I}
        {México, D.F}{}{}
        \cvitem{}{Implementación parcial de un sistema automatizado para la generación y aplicación de exámenes}
    \cvitem{dirección}{Alfonso Martinez Martinez}
    \cvitem{funciones}{
        \begin{itemize}
        \item Depuración de la base de código Code Igniter presente
        \item Limpieza del modelo de datos MySQL
        \item Inicio de la codificación de sub-aplicaciones para altas, bajas y cambios
        \end{itemize}
    }

\subsection{Desarrollo independiente}


\subsection{Talleres Impartidos}

\cventry{2009}{Talleres Intertrimestrales Otoño \emph{(Summer Workshop)}}{CEUAMI, UAM
Iztapalapa}{México, D.F}{}{Introducción a Haskell \emph{(Introduction to the Haskell programming language)}} %% FIXME buscar reconocimiento
\cventry{2009}{Festival Latinoamericano de Instalación de Software Libre \emph{(Latin American free software install fest)}}{UAM Iztapalapa}{México, D.F}{}{Interacción Java/Python \emph{(Java/Python Interaction)}}
\cventry{2007}{XIV Semana de Ingeniería Eléctrica \emph{(XIV Electrical Engineering Week)}}{UAM Iztapalapa}{México, D.F}{}{Introducción al lenguaje de programación Python \emph{(Introduction to the Python Programming Language)}}
\cventry{2007}{Talleres Intertrimestrales Primavera \emph{(Spring Workshop)}}{CEUAMI, UAM Iztapalapa}{México, D.F}{}{Programación básica de videojuegos en Python \emph{(Introduction to Python videogame programming)}}


\subsection{Conferencias}
\cventry{2006}{Congreso Nacional de Software Libre \emph{(National Free Software Congress)}}{ESIME Zacatenco, IPN}{México,
D.F}{}{Verificación de Memoria en Linux con Valgrind \emph{(Memory verification in Linux using Valgrind)}}
\cventry{2005}{Congreso Nacional de Software Libre}{UAM Iztapalapa}{México,
D.F}{}{KDE: Diseño e implementación de aplicación en 3D extendida por
Python (\emph{KDE: Design and Implementation of a Python extended 3D app})}

\section{Estudios de Licenciatura}
\cventry{2005--2011}{Computación}{UAM Iztapalapa}{México, D.F}{}{}
\cvitem{optativas}{Botánica I (ficología), Genética General, Algoritmos Distribuídos,
    Inteligencia Artificial, Lenguajes de Programación}
\section{Proyecto de Investigación}
\cvitem{título}{\emph{Políticas de Manejo de Memoria para una Aplicación de Gráficas}}
\cvitem{asesor}{Elizabeth Pérez Cortés}

\cvitem{descripción}{
    Simulamos el comportamiento de la ejecución de programas
    suponiendo la existencia de un segundo nivel de
    almacenamiento, empleando algoritmos dependientes de la estructura de los
    datos en el programa para planificar el manejo de memoria.
    Se buscaron mejoras en la tasa de transferencia entre niveles de
    almacenamiento.
}

\section{Servicio Social}
\cvitem{2010}{\emph{Verificación automatizada de modelos de circuito básicos}}


\cvitem{asesor}{Oscar Yáñez Suárez}

\cvitem{descripción}{
    Implementamos un compilador simple para traducir circuitos RL de lenguaje
    SPICE a gráficas con aristas coloreadas. Las gráficas fueron
    comparadas para buscar isomorfismo, y de esta manera determinar si el
    circuito presentado por un estudiante corresponde a la referencia del
    instructor.
}

\section{Habilidades}

\section{Varios}

\subsection{Organización de Eventos}
\cventry{2007}{Seminario de Ciencias de la Computación}{CEUAMI, UAM Iztapalapa}{México, D.F}{}{}
\cventry{2007}{Semana de talleres de Invierno}{CEUAMI, UAM Iztapalapa}{México, D.F}{}{}
\cventry{2006}{XIII semana estudiantil de Ingeniería Eléctrica}{UAM Iztapalapa}{México, D.F}{}{}

\end{document}
